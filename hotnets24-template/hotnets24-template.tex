\documentclass[sigconf,10pt,anonymous,review,nonacm]{acmart}
\settopmatter{printfolios=true,printccs=false,printacmref=false}

\title{An Amazing HotNets 2024 Paper}

\author{Behnaz Arzani}
\affiliation{%
  \institution{Microsoft Research}%
  \city{Redmond, WA}
  \country{USA}%
}
\author{Nate Foster}
\affiliation{%
  \institution{Cornell University}%
  \city{Ithaca, NY}
  \country{USA}%  
}  

\begin{abstract}
The  body of this document contains a copy of the HotNets '24 call for papers to illustrate how a standard page should be formatted. As an example of a citation, everybody loves Cerf and Kahn's classic work on the TCP/IP protocol suite~\cite{cerf-kahn-1974}. 
\end{abstract}

\begin{document}

\maketitle

\section{Call for Papers}

The 23rd ACM Workshop on Hot Topics in Networks (HotNets 2024) will bring together researchers to engage in a lively debate on the theory and practice of computer networking and networked systems.  HotNets provides a venue to discuss innovative ideas and debate future research agendas for the field.

We invite researchers and practitioners to submit short position papers. We encourage submissions that identify fundamental open questions, advocate for new approaches, offer constructive critiques of the state of networking research, re-frame or debunk existing work, report unexpected early results from a deployment, report promising but unproven ideas, or propose new evaluation methods. Authors need not support their ideas with full evaluations; we consider well-reasoned arguments or preliminary evaluations sufficient support for the possibility of the paper’s claims in HotNets.

We seek early-stage work, where the authors can benefit from community feedback. An ideal submission has the potential to open a line of inquiry for the community that results in multiple conference papers in related venues (SIGCOMM, NSDI, CoNEXT, SOSP, OSDI, MobiCom, MobiSys, etc.), rather than a single follow-on conference paper. We strongly discourage ``conference papers in the miniature,'' and the program committee will explicitly favor early work and papers likely to stimulate reflection and discussion over something that is ready for publication at a conference. Finished work that fits in a short paper is likely a better fit with the short-paper tracks at CoNEXT or IMC.

HotNets takes a broad view of networking research. The scope includes new ideas that relate to (but not limited to) mobile, wide-area, data-center, home, and enterprise networks and that use a variety of link technologies (e.g. wired, wireless, optical, visual, and acoustic), as well as social networks and network architecture. HotNets encompasses all aspects of networks, which include (but is not limited to) packet-processing and transmission hardware and software, virtualization, mobility, provisioning and resource management, performance, energy consumption, sustainability, topology, robustness and security, measurement, diagnosis, verification, privacy, economics and evolution, theory, usability, machine learning, and interactions with applications.

The program committee will select position papers based on originality, likelihood to stimulate insightful discussion, and technical merit. Accepted papers will be posted online prior to the workshop and will be published in the ACM Digital Library to facilitate wide dissemination of the ideas discussed at the workshop.

\section{Call for PhD Student Abstracts}

To spotlight recent work by PhD students, who will shape the future of the networking community, the HotNets 2024 workshop will also include short talks by graduating PhD students (those who expect to receive their PhD degree by Summer 2025). Talks should focus on articulating a future research agenda for the field. Interested PhD students should send a title and a short (maximum 1000 words) abstract of their proposed talk to the HotNets 2024 PC chairs at \texttt{bearzani@microsoft.com} and \texttt{jnfoster@cs.cornell.edu} no later than July 15th 2024 (AoE).

\section{Policy on Concurrent Submissions}

Concurrent submissions to HotNets 2024 and any other peer-reviewed venue that cover the same work (differences in the degree of detail given the two venues’ length limits notwithstanding) are prohibited, and will result in the immediate rejection of the HotNets submission in question. “Concurrent” means any other peer-reviewed venue whose reviewing period (i.e., between submission and notification) overlaps with that of HotNets. The ``same work'' means, for example, a submission overlapping significantly in content with a conference submission. However, a position paper submitted to HotNets (e.g., that reflects broadly on the state of some aspect of the field, adopts a position as to how the field should move forward, or articulates a broad avenue of future work) will not be considered the ``same work'' as a conference-length paper on a specific system that addresses a point under the broad umbrella covered by the position paper. Authors with questions about this policy should contact the PC co-chairs before submitting.

\section{Ethical Concerns}

As part of the submission process, authors must attest that their work complies with all applicable ethical standards of their home institution(s), including, but not limited to privacy policies and policies on experiments involving humans. Note that submitting research for approval by one’s institution’s ethics review body is necessary, but not sufficient---in cases where the PC has concerns about the ethics of the work in a submission, the PC will have its own discussion of the ethics of that work. The PC takes a broad view of what constitutes an ethical concern, and authors agree to be available at any time during the review process to rapidly respond to queries from the PC chairs regarding ethical standards.

\section{Workshop Participation}

HotNets attendance is limited to roughly 90 people in person to facilitate a lively discussion (potentially with additional virtual participants). In-person invitations will be allocated first to one author of each paper, HotNets organizers and committee members, and conference sponsors. We require the paper presentations to be in-person (if authors have any concerns about this, they are welcome to contact the PC chairs). To promote an inclusive workshop, HotNets will also make a limited number of open registration slots for in-person attendance available to the community.

\section{Conclusion}

Thank you for submitting to HotNets '24. We look forward to reading your submissions!

\section*{Acknowledgements}

This document is based on the template used for HotNets '23, modifed to use the standard ACM article class.

\bibliographystyle{ACM-Reference-Format} 
\bibliography{hotnets24-template}

\end{document}
